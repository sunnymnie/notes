\documentclass[10pt,twocolumn]{../notes}
\title{Atomic Habits}
\begin{document}
\maketitle
\tableofcontents
\section{The Fundamentals}
\Advice To make earth-shattering improvements, do not seek for massive changes, but change by just making small improvements on a daily basis
\Quote If you want to predict where you’ll end up in life, all you have to do is follow the curve of tiny gains or tiny losses, and see how your daily choices will compound ten or twenty years down the line

\Advice Be patient with your habits
\Fact The effect of small habits are delayed. You don’t immediately see good results from good small habits, and neither bad results from small bad habits. But, overtime, they compound to something much larger
\Quote Your outcomes are a lagging measure of your habits
\Fact Why people stop building positive habits: They see that they’re failing to make a tangible result and decide to quit
\Quote It is not until months or years later that we realize the true value of the previous work we have done
\Quote Small changes often appear to make no difference until you cross a critical threshold. The most powerful outcomes of any compounding process are delayed. You need to be patient

\Image{the-plateau}{the plateau of latent potential}

\Advice Focus not only on making good habits, but also removing bad ones
\Quote Habits are a double-edged sword. They can work for you or against you
\begin{List}{Positive Compounding}
  \item Productivity compounds: Accomplishing one extra task every day or automating an old task every day can lead to a lot over an entire career.
  \item Knowledge compounds: Learning one new idea every day not only makes you know more, but allows you to see old ideas through new perspectives.
  \item Relationships compound: Helping others more and being nicer in each interaction can result in a network of strong connections over time.
\end{List}

\begin{List}{Negative Compounding}
\item Stress compounds: Stressing every day can lead to serious health issues
\item Negative thoughts compound: The more you think of yourself or others negatively, you will spiral uncontrollably towards that direction.
\end{List}

\Advice Focus on systems rather than setting goals.
\Fact Society worries about setting specific, actionable goals, but the real results lie when you focus on the system.
\Quote You do not rise to the level of your goals. You fall to the level of your systems

\section{How Your Habits Shape Your Identity (and Vice Versa)}
\Image{opi-chart}{Three layers of behaviour change}
\Bold{Outcomes:} Concerned with results (ex: publishing book, general goals)
\Bold{Process:} Concerned with changing your habits and systems (ex: new gym routine)
\Bold{Identity:} Changing your beliefs, worldview, self-image, and judgements

\Advice Focus on who you wish to become rather than what you want to achieve
\Example Being offered a cigarette and saying “No thanks. I’m trying to quit” vs. “No thanks. I’m not a smoker.” Small difference but one is a goal and the other is a shift in identity
\Quote The ultimate form of intrinsic motivation is when a habit becomes part of your identity
\Fact Once your pride gets involved in your habit, you’ll fight hard to maintain them
\Quote True behaviour change is identity change. You might start a habit because of motivation, but the only reason you’ll stick with one is that it becomes part of your identity
\Example The goal is not to read a book, the goal is to become a reader.
\Example The goal is not to run a marathon, the goal is to become a runner.
\Example The goal is not to learn an instrument, the goal is to become a musician.
\Fact The reason why your positive habits fail to stick is because they conflict with your current identity
\Fact Feedback loops: Your habits shape your identity, and your identity shapes your habits

\Advice Know what result you want, work backward to find what type of person could get those results. Work to become that person by continuously accumulating small habits
\begin{Olist}{Two step process to changing your identity}
  \item Decide the type of person you want to be
  \item Prove it to yourself with small wins
\end{Olist}
\Quote The most effective way to change your habits is to focus not on what you want to achieve, but on who you wish to become
\Quote Every action you take is a vote for the type of person you wish to become. No single instance will transform your beliefs, but as the votes build up, so does the evidence of your new identity
\Quote The most practical way to change who you are is to change what you do

\Image{habit-stages-chart}{the science of how habits work}
\begin{Olist}{}
  \item Cue: Triggers brain to initiate a behaviour
  \item Craving: Motivational force behind every habit. Every craving is linked to a desire to change your internal state
  \item Response: The actual habit performed. If friction is low and the reward is great, the response will be carried out
  \item Reward: End goal of every habit. “The cue is about noticing the reward. The craving is about wanting the reward. The response is about obtaining the reward
\end{Olist}
\Fact We chase rewards because they satisfy us and because they teach us that these actions are worth remembering in the future.
\Fact If a behaviour is insufficient in any of the four stages, it will not become a habit

\section{The 1st Law: Make it Obvious}
\Advice Create a habits scorecard to observe what is going on
\Quote This is one of the most surprising insights about our habits: you don’t need to be aware of the cue for a habit to begin. You can notice an opportunity and take action without dedicating conscious attention to it. This is what makes habits useful.
\Fact Many of our habits are performed subconsciously
\Quote Once our habits become automatic, we stop paying attention to what we are doing.
\Quote The process of behaviour change always starts with awareness. You need to be aware of your habits before you can change them.
\begin{Olist}{How to create a habits scorecard}
  \item List out all of your daily habits
  \item Looking at each habit, ask yourself if that habit is a good, bad, or neutral habit. Mark each habit on the list as such.
  \item If a habit is hard to determine with step 2, ask yourself: “Does this behaviour help me become the type of person I wish to be? Does this habit cast a vote for or against my desired identity?”
\end{Olist}
\Advice Create an implementation intention: “I will BEHAVIOUR at TIME in LOCATION”
\Example Studying. I will study Spanish for twenty minutes at 6 p.m. in my bedroom
\Example Exercise. I will exercise for one hour at 5 p.m. in my local gym
\Example Marriage. I will make my partner a cup of tea at 8 a.m. in the kitchen
\Fact The two most important cues that can trigger a habit are time and place
\Quote People who make a specific plan for when and where they will perform a new habit are more likely to follow through

\Advice Use Habit Stacking to build new habits
\Quote One of the best ways to build a new habit is to identify a current habit you already do each day and then stack your new behaviour on top. This is called habit stacking
\Quote Rather than pairing your new habit with a particular time and location, you pair it with a current habit

\Advice Create a habit using the habit stacking formula: “After CURRENT HABIT, I will NEW HABIT”.
\Example Meditation. After I pour my cup of coffee each morning, I will meditate for one minute.
\Example Exercise. After I take off my work shoes, I will immediately change into my workout clothes.
\Example Gratitude. After I sit down to dinner, I will say one thing I’m grateful for that happened today.
\Example Marriage. After I get into bed at night, I will give my partner a kiss.
\Example Safety. After I put on my running shoes, I will text a friend or family member where I am running and how long it will take.
\Fact Once you have a basic structure, you can begin creating longer stacks by chaining your small habits together using the Diderot Effect.
\Example Case 1: After I pour my morning cup of coffee, I will meditate for sixty seconds. After I meditate for sixty seconds, I will write my to-do list for the day. After I write my to-do list for the day, I will immediately begin my first task.
\Example Case 2: After I finish eating dinner, I will put my plate directly into the dishwasher. After I put my dishes away, I will immediately wipe down the counter. After I wipe down the counter, I will set out my coffee mug for tomorrow morning.

\Advice Create general habit stacks:
\Example Exercise. When I see a set of stairs, I will take them instead of using the elevator.
\Example Social skills. When I walk into a party, I will introduce myself to someone I don’t know yet.
\Example Finances. When I want to buy something over \$100, I will wait twenty-four hours before purchasing.
\Example Healthy eating. When I serve myself a meal, I will always put veggies on my plate first.
\Example Minimalism. When I buy a new item, I will give something away. (“One in, one out.”)
\Example Mood. When the phone rings, I will take one deep breath and smile before answering.
\Example Forgetfulness. When I leave a public place, I will check the table and chairs to make sure I don’t leave anything behind.
\Fact Habit stacking works best when the cue is highly specific and immediately actionable.
\Example BAD: When I take a break for lunch, I will do ten push-ups.
\Example GOOD: When I close my laptop for lunch, I will do ten push-ups next to my desk.

\Advice Design your environment such that cues of preferred habits are obvious.
\Fact When the cues that spark a habit are subtle or hidden, they are easy to ignore.
\Example If you want to remember to take your medication each night, put your pill bottle directly next to the faucet on the bathroom counter.
\Example If you want to practice guitar more frequently, place your guitar stand in the middle of the living room.
\Example If you want to remember to send more thank-you notes, keep a stack of stationery on your desk.
\Example If you want to drink more water, fill up a few water bottles each morning and place them in common locations around the house.
\Fact Gradually, your habits become associated not with a single trigger but with the entire context surrounding the behaviour. The context becomes the cue.

\Advice Create a separate space for work, study, exercise, entertainment, and cooking.
\Fact It is easier to build new habits in a new environment because you are not fighting against old cues.
\Quote Every habit should have a home.
\Quote A stable environment where everything has a place and a purpose is an environment where habits can easily form.

\Advice Cut bad habits off at the source by changing your environment or eliminating the cues.
\Quote So, yes, perseverance, grit, and willpower are essential to success, but the way to improve these qualities is not by wishing you were a more disciplined person, but by creating a more disciplined environment.
\Fact A habit that has been encoded in the mind is ready to be used whenever the relevant situation arises.
\Fact People with high self-control tend to spend less time in tempting situations. It’s easier to avoid temptation than resist it.
\Fact You can break a habit, but you’re unlikely to forget it. Once the mental grooves of habit have been carved into your brain, they are nearly impossible to remove entirely—even if they go unused for quite a while.
\begin{Examplelist}{}
  \item If you can’t seem to get any work done, leave your phone in another room for a few hours.
  \item If you’re continually feeling like you’re not enough, stop following social media accounts that trigger jealousy and envy.
  \item If you’re wasting too much time watching television, move the TV out of the bedroom.
  \item If you’re spending too much money on electronics, quit reading reviews of the latest tech gear.
  \item If you’re playing too many video games, unplug the console and put it in a closet after each use.
\end{Examplelist}

\section{The 2nd Law: Make it Attractive}

\Advice If you want to increase the odds that a behaviour will occur, then you need to make it attractive.
\Fact It is the anticipation of a reward—not the fulfillment of it—that gets us to take action. The greater the anticipation, the greater the dopamine spike.
\Quote Whenever you predict that an opportunity will be rewarding, your levels of dopamine spike in anticipation. And whenever dopamine rises, so does your motivation to act.

% Start
\Advice Combine habit stacking + temptation bundling:
\Quote Temptation bundling works by linking an action you want to do with an action you need to do.
\begin{Olist}{HOW}
  \item After CURRENT HABIT, I will HABIT I NEED.
  \item After HABIT I NEED, I will HABIT I WANT.
\end{Olist}
\Example Case 1: If you want to read the news, but you need to express more gratitude: After I get my morning coffee, I will say one thing I’m grateful for that happened yesterday (need). After I say one thing I’m grateful for, I will read the news (want).
\Example Case 2: If you want to watch sports, but you need to make sales calls: After I get back from my lunch break, I will call three potential clients (need). After I call three potential clients, I will check ESPN (want).
\Example Case 3: If you want to check Facebook, but you need to exercise more: After I pull out my phone, I will do ten burpees (need). After I do ten burpees, I will check Facebook (want).

\Advice Join a culture where (1) your desired behaviour is the normal behaviour and (2) you already have something in common with the group.
\Fact Behaviours are attractive when they help us fit in.
\Fact The culture we live in determines which behaviours are attractive to us.
\Fact The normal behaviour of the tribe often overpowers the desired behaviour of the individual. Most days, we’d rather be wrong with the crowd than be right by ourselves.
\Fact We tend to imitate the habits of three social groups: the close (family and friends), the many (the tribe), and the powerful (those with status and prestige).
\Quote One of the most effective things you can do to build better habits is to join a culture where your desired behaviour is the normal behaviour.
\Fact People imitate the habits of the close, the many, and the powerful
\Fact Whenever we are unsure how to act, we look to the group to guide our behaviour.
\Fact We try to copy the behaviour of successful people because we desire success ourselves.
\Fact If a behaviour can get us approval, respect, and praise, we find it attractive.

\Advice Associate hard habits with positive experience by reframing your habits to highlight their benefits rather than their drawbacks

\begin{Examplelist}{Case: Exercise}
  \item NOT: I need to go run in the morning.
  \item DO: It’s time to build endurance and get fast.
\end{Examplelist}

\Quote Habits are attractive when we associate them with positive feelings and unattractive when we associate them with negative feelings. Create a motivation ritual by doing something you enjoy immediately before a difficult habit
\Fact Every habit has a deeper underlying motive other than surface desires.
\Fact Your habits are modern-day solutions to ancient problems.
\Example Find love and reproduce = using Tinder
\Example Connect and bond with others = browsing Facebook
\Example Win social acceptance and approval = posting on Instagram
\Example Reduce uncertainty = searching on Google
\Example Achieve status and prestige = playing video games
\Fact Your current habits are not necessarily the best way to solve the problems you face; they are just the methods you learned to use.
\Example Smoking to relieve stress vs running to reduce anxiety
\Quote Once you associate a solution with the problem you need to solve, you keep coming back to it.

\section{The 3rd Law: Make it Easy}
\Advice Focus on taking action rather than planning and strategizing.
\Quote The most effective form of learning is practice, not planning.
\Quote We are so focused on figuring out the best approach that we never get around to taking action.
\Fact Planning, strategizing, and learning makes it feel like you’re making progress real when you’re not.
\Quote You don’t want to merely be planning. You want to be practicing.
\Quote Habit formation is the process by which a behaviour becomes progressively more automatic through repetition.
\Quote The amount of time you have been performing a habit is not as important as the number of times you have performed it.
\Image{habit-line}{the habit line}

\Advice Make your habits convenient
\How Increase the friction associated with bad behaviours and reduce the friction associated with good habits. When friction is high, habits are difficult, and vice versa.
\Quote Human behaviour follows the Law of Least Effort. We will naturally gravitate toward the option that requires the least amount of work.
\Fact The less energy a habit requires, the more likely it is to occur
\Quote Create an environment where doing the right thing is as easy as possible.
\Quote Reduce the friction associated with good behaviours. When friction is low, habits are easy.

\Advice Set up your environment so the next time you perform a good habit will be very easy and very hard to do a bad habit.
\Quote Prime your environment to make future actions easier.

\Advice Use the two minutes rule
\Quote The Two-Minute Rule states, “When you start a new habit, it should take less than two minutes to do.”
\begin{Examplelist}{}
\item “Read before bed each night” becomes “Read one page.”
\item “Do thirty minutes of yoga” becomes “Take out my yoga mat.”
\item “Study for class” becomes “Open my notes.”
\item “Fold the laundry” becomes “Fold one pair of socks.”
\item “Run three miles” becomes “Tie my running shoes.”
\end{Examplelist}

\Advice Enforce the two minute rule if you believe you are tricking yourself. Force yourself to quit after exactly two minutes if you don’t feel like working on it.
\Quote The more you ritualize the beginning of a process, the more likely it becomes that you can slip into the state of deep focus that is required to do great things.
\Quote Standardize before you optimize. You can’t improve a habit that doesn’t exist.
\Quote Instead of trying to engineer a perfect habit from the start, do the easy thing on a more consistent basis.

\Advice Consider habituating an easier habit if it leads to working on a more difficult habit
Make your habit stepping into the bus rather than the gym workout.
\Quote Habits can be completed in a few seconds but continue to impact your behaviour for minutes or hours afterward.
\Quote Many habits occur at decisive moments—choices that are like a fork in the road—and either send you in the direction of a productive day or an unproductive one.

\Advice Make doing bad habits difficult by using a commitment device
\Quote A commitment device is a choice you make in the present that locks in better behaviour in the future.
\begin{Examplelist}{}
\item Buy food in individual packages rather than bulk size to prevent overeating
\item Ask to be added to the ban list at casinos to prevent gambling sprees
\item Buy a device that shuts internet off at 10pm to sleep early
\end{Examplelist}
\Quote Onetime choices—like buying a better mattress or enrolling in an automatic savings plan—are single actions that automate your future habits and deliver increasing returns over time.

\Advice Find ways to automate your habits whenever possible
\Quote Using technology to automate your habits is the most reliable and effective way to guarantee the right behaviour.

\section{The 4th Law: Make it Satisfying}

\Advice Add immediate pleasure to good habits and immediate pain to bad habits.
\Fact “We are more likely to repeat a behaviour when the experience is satisfying.”
\Fact ”The human brain evolved to prioritize immediate rewards over delayed rewards.”
\Fact “The Cardinal Rule of Behaviour Change: What is immediately rewarded is repeated. What is immediately punished is avoided.”
\Quote To get a habit to stick you need to feel immediately successful—even if it’s in a small way.
\Quote The first three laws of behaviour change—make it obvious, make it attractive, and make it easy—increase the odds that a behaviour will be per formed this time. The fourth law of behaviour change—make it satisfying—increases the odds that a behaviour will be repeated next time.

\Advice When selecting rewards for habits, do not select rewards that run counter to what the habit is about.
\begin{Examplelist}{}
\item Do NOT reward buying a new jacket if habit is to save money
\item CAN reward buying a new jacket if habit is to read more.
\item Do NOT reward eating ice-cream if habit is to lose weight
\end{Examplelist}
\Quote Incentives can start a habit. Identity sustains a habit.

\Advice Use immediate pleasure to start a habit, but slowly it would be unnecessary and habit will become self sustaining

\Advice Track your habits
\Quote One of the most satisfying feelings is the feeling of making progress.
Moving paperclip from one jar to another for each sales call you make
\Quote A habit tracker is a simple way to measure whether you did a habit—like marking an X on a calendar.

\Advice Tracking habits should be as automated and simple as possible
\Fact It is difficult to start a habit, it is even more difficult to start a habit and start a habit tracking habit at the same time.
\begin{Examplelist}{}
  \item Credit card statement tracks how often you go out to eat
  \item Fitbit tells you how many steps you take and how long you sleep
\end{Examplelist}{}

\Advice Only manually track your most important habits.

\Advice Immediately measure and mark your habit after finishing a session.
\Quote Don’t break the chain. Try to keep your habit streak alive.
\Quote Never miss twice. If you miss one day, try to get back on track as quickly as possible.

\Advice Get an accountability partner to make doing bad habits unsatisfying
\Fact We are less likely to repeat a bad habit if it has immediate painful consequences.
\Quote In general, the more local, tangible, concrete, and immediate the consequence, the more likely it is to influence individual behaviour.
\Quote An accountability partner can create an immediate cost to inaction. We care deeply about what others think of us, and we do not want others to have a lesser opinion of us.
\Quote A habit contract can be used to add a social cost to any behaviour. It makes the costs of violating your promises public and painful.
\Quote Knowing that someone else is watching you can be a powerful motivator.
Advanced Tactics: How to Go from Being Merely Good to Being Truly Great

\Advice The Goldilocks Rule states that humans experience peak motivation when working on tasks that are right on the edge of their current abilities.

\Image{the-goldilocks-zone}{the goldilocks zone}

\Quote The greatest threat to success is not failure but boredom.
\Quote As habits become routine, they become less interesting and less satisfying. We get bored.
\Quote Anyone can work hard when they feel motivated. It’s the ability to keep going when work isn’t exciting that makes the difference.
\Fact Really successful people feel the same lack of motivation as everyone else, however, they still find a way to practice despite the boredom.
\Quote Professionals stick to the schedule; amateurs let life get in the way.

\Image{mastering-a-field}{mastering a field}

\Advice You need to layer improvements on top of one another.
\Quote Right when things are starting to feel automatic and you are becoming comfortable—that you must avoid slipping into the trap of complacency.

\Advice Engage in reflection and review to make sure you are still improving
\begin{Examplelist}{Every summer: }
  \item What are the core values that drive my life and work?
  \item How am I living and working with integrity right now?
  \item How can I set a higher standard in the future?
\end{Examplelist}
\begin{Examplelist}{Every new year: }
  \item What went well this year?
  \item What didn’t go so well this year?
  \item What did I learn?
\end{Examplelist}
\Quote Reflection and review is a process that allows you to remain conscious of your performance over time.

\Advice Keep your identity small
\Quote The tighter we cling to an identity, the harder it becomes to grow beyond it.
\Quote The more you let a single belief define you, the less capable you are of adapting when life challenges you.


\end{document}
