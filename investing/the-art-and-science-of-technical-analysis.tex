\documentclass[10pt,twocolumn]{article}
\title{The Art and Science of Technical Analysis}
\begin{document}
\maketitle
\tableofcontents

\part{The Foundation of Technical Analysis}
\section{The Trader's Edge}
Use both \textit{technical} and \textit{fundamental} factors when trading.

Price represents the end product of the analysis and decision making of all market participants, and believe that a careful analysis of price movements can sometimes reveal areas of market imbalance that can offer opportunities for superior risk- adjusted profits.
\subsection{Defining a trading edge}
\begin{itemize}
  \item Excellent execution, risk management, discipline, and proper psychology are all important elements of a good trading plan, but it is all futile if the trading system does not have a positive expectancy.
  \item In all cases, the trading problem reduces to a matter of identifying when a statistical edge is present in the market, acting accordingly, and avoiding market environments that are more random. To do this well, it is essential to have a good understanding of how markets move and also some of the math behind expectancy and probability theory.

\end{itemize}

\subsubsection{Expected Value}
\begin{itemize}
  \item A positive expectancy results when the trader successfully identifies those moments where markets are slightly less random than usual, and places trades that are aligned with the slight statistical edges present in those areas.
  \item If you are not trading with a statistical advantage over the market, everything else is futile.
  \item From a statistical standpoint, the definition of an edge is simple: can you properly identify entry and exit points in the market so that, over a large sample size, the sum of the profit and loss (P\&L) from your winning trades is greater than the sum of your losing trades? The question then becomes: how do you find, develop, refine, and maintain an edge? There are many answers to that question; this book shows one possible path.
\end{itemize}

\subsubsection{Where Does the Edge Come From?}
\begin{itemize}
  \item In an idealized, mathematical random walk world, price would have no memory of where it has been in the past; but in the real world, prices are determined by traders making buy and sell decisions at specific times and prices. When markets revisit these specific prices, the market does have a memory, and we frequently see nonrandom action on these retests of important price levels. People remember the hopes, fears, and pain associated with price extremes. In addition, most large-scale buying follows a more or less predictable pattern: traders and execution algorithms alike will execute part of large orders aggressively, and then will wait to allow the market to absorb the action before resuming their executions. The more aggressive the buyers, the further they will lift offers and the less they will wait between spurts of buying. This type of action, and the memory of other traders around previous inflections, creates slight but predictable tendencies in prices.
  \item The conclusion is logical and unavoidable: buying and selling pressure must, by necessity, leave patterns in the market. Our challenge is to understand how psychology can shape market structure and price action, and to find places where this buying and selling pressure creates opportunities in the form of nonrandom price action.
\end{itemize}

\subsubsection{The Holy Grail}
\textbf{\textit{Every edge we have, as technical traders, comes from an imbalance of buying and selling pressure.}}

If we realize this and if we limit our involvement in the market to those points where there is an actual imbalance, then there is the possibility of making profits. We can sometimes identify these imbalances through the patterns they create in prices, and these patterns can provide actual points around which to structure and execute trades.


\subsection{Finding and developing your edge}
The process of developing and refining your edge in the market is exactly that: an ongoing process.

\subsubsection{Why small traders can make money}
\begin{itemize}
  \item you are not mandated to have any specific exposures.
  \item you will likely experience few, if any, liquidity or size issues; your orders will have a minimal (but still very real) impact on prices.
  \item you can move freely among currencies, equities, futures, and options, using outright or spread strategies as appropriate.
  \item you are free to target a time frame that is not interesting to many institutions and not accessible to some.
\end{itemize}

\subsection{General principles of chart reading}
Though it is possible to trade by focusing on simple chart patterns, this approach also misses much of the richness and depth of analysis that are available to a skilled chart reader.

Once traders learn to read the message of the market, they can understand the psychological tone and the balance of buying and selling pressure at any point.

\subsubsection{Chart setup}

When it comes to chart setup, there is no one right way, but I will share my approach. Everything I do comes from an emphasis on clarity and consistency. Clean charts put the focus where it belongs: on the price bars and the developing market structure. Tools that highlight and emphasize the underlying market’s structure are good; anything that detracts from that focus is bad. When you see a chart, you want the price bars (or candles) to be the first and most important thing your eye is drawn to; any calculated measure is only a supplement or an enhancement. Consistency is also very important, for two separate reasons. First, consistency reduces the time required to orient between charts. It is not unusual for me to scan 500 charts in a single sitting, and I can effectively do this by spending a little over a second on each chart. This is possible only because every one of my charts has the same layout and I can instantly orient and drill down to the relevant information. Consistency is also especially important for the developing trader because part of the learning process is training your eye to process data a certain way.

\textbf{TODO: JUPYTER and JAVA: Set up program to download from binance charts and allow me to skim the charts very quickly}

Use log scale charts when:
\begin{enumerate}
  \item There is a greater than 100\% price increase on a chart
  \item Chart is showing more than 2 years of data
\end{enumerate}

\subsubsection{Choosing Time Frames}
Discretionary traders must clearly choose and define the time frame within which they will trade, and this choice of time frames is tied into deeper questions of personality and trading style. Most of the trading ideas and principles we examine in this book can be applied to all markets and all time frames, with some adjustments, but most traders will probably find themselves best suited to a specific set of markets and time frames. Traders switching time frames or asset classes will usually undergo a painful adjustment period while they figure out how to apply their tools in the new context.

The rule of consistency also applies to choice of time frames. Once you have settled on a trading style and time frame, be slow to modify it unless you have evidence that it is not working. This story will be told with the most clarity and power in a consistent time frame. In addition, if you catch yourself wanting to look at a time frame you never look at while you are in a losing trade, be very careful. This is often a warning of an impending break of discipline.

\begin{itemize}
  \item In a scheme like this, the primary time frame of focus is called the trading time frame (TTF). A higher time frame (HTF) chart provides a bigger- picture perspective, while a lower time frame (LTF) chart is usually used to find precise entry points. Other variations, with up to five or six charts, are possible, and there are many traders who use only a pair of charts. Last, though the term time frame seems to imply that the x-axis of the chart will be a time scale (minutes, hours, days, etc.), the same proportional relationships can be applied to tick, volume, or any other activity- based axis scale on the x-axis.
  \item In general, time frames should be related to each other by a factor of 3 to 5. There is no magic in these ratios, but the idea is that each time frame should provide new information without loss of resolution or unnecessary repetition.
\end{itemize}

\subsubsection{Bars, Candles, or Other Choices}
The main advantage of bar charts is that they can be cleaner visually and it is usually possible to fit more data in the same space because bars are thinner than candles.


\subsection{Indicators}
There is certainly no one right way to set up or use indicators, but, here again, consistency is paramount. Few traders find success by constantly switching between indicators. There is no holy grail or combination of tools that will lead to easy trading profits.

One other important point is that you must intimately understand the tools you use. Know how they will react to all market conditions, and know what they are saying about the market structure and price action at any time. Focus on tools that highlight and emphasize important elements of market structure, because your main focus should be on the price bars themselves.

\subsection{The two forces: toward a new understanding of market action}
In this book, price action simply means how markets usually move, which, frankly, is, usually randomly. Be clear on this point: markets are usually random and most of the patterns markets create are also random. However, we can sometimes identify spots where price movement is something less than random and is somewhat more predictable, and these less-than-random spots may offer profitable trading opportunities.

Price action is the term used to describe the market’s movements in a dynamic state. Price action creates market structure, which is the static record of how prices moved in the past.

In the case of actual price action, we would look at elements such as: How does the market react after a large movement in one direction? If aggressive sellers are pressing the market lower, what happens when they relax their selling pressure? Does the market bounce back quickly, indicating that buyers are potentially interested in these depressed prices, or does it sit quietly, resting at lower levels? How rapidly are new orders coming into the market? Is trading one-directional, or is there more two-way, back-and-forth trading? Are price levels reached through continuous motion, or do very large orders cause large jerks in prices? All of these elements, and many more, combine to describe how the market moves in response to order flow and a myriad of competing influences.

I propose a simpler model: that market action appears to be the result of two interacting forces: a motive force that attempts to move price from one level to another and a resistive force that opposes the motive force. These forces represent the sum of all analysis and decision making at any one time.

The normal state of existence in most markets most of the time is equilibrium. The two forces are in balance. Buyers and sellers have no sharp disagreement over price; the market may drift around a central value, but there are no large trends or price changes. Market action in this environment is highly random; if we were to analyze this type of action statistically, we would find that it conforms very closely to a random walk model. This is also precisely the type of environment that technically motivated traders must strive to avoid, as there can be no enduring statistical edge in a randomly driven market.

Markets in this state of equilibrium will have varying degrees of liquidity and ability to absorb large orders. Eventually, there is a failure of liquidity on one side, and the market makes a sudden, large movement in one direction. Perhaps this movement is in response to new information coming into the market, or it can simply be a result of a random price movement setting off further movement in the same direction. No matter the reason behind the movement, in theoretical terms, the motive force has, at least temporarily, overcome the resistive force. In the parlance of technical analysis, this type of sharp movement is called an impulse move or a momentum move.

From this point, there are basically two options.
\begin{enumerate}
  \item In many cases, the resistive force is quickly able to overcome the motive force, and the market finds balance again.
  \begin{itemize}
    \item This may be at a new level, or prices may immediately retrace their course and return to the preshock levels.
    \item Psychologically, market participants have chosen to view this large price movement as a temporary aberration, and new liquidity comes into the market that will dampen any future distortion.
  \end{itemize}
  \item However, it is possible that the large price spike will lead to continued movement in the same direction.
  \begin{itemize}
    \item In this scenario, a feedback loop develops where the market makes a large movement, which, in turn, provokes another large price movement, and the market trends.
  \end{itemize}
\end{enumerate}

In most cases, the market structure of this trending movement will be a series of directional moves alternating with nondirectional periods in which the market essentially rests and absorbs the previous move. In the bigger picture, the motive force has overcome the resistive force, but there is still a subtle interplay of balance and imbalance on shorter time frames. Prices trend because of an imbalance of buying and selling pressure. (This is often, but not always, indicative of nonrandom action, as trends exist in completely random data.) Once prices are trending, at some point they will have moved far enough that the resistive force is once again able to balance the motive force, and the market again finds a new balance.

This interplay of motive and resistive forces, from a very high-level perspective, is the essence of price action and the root of technical analysis. The patterns we see in the market are only reflections of the convictions of buyers and sellers. They are useful because we can see them, trade them, and use them to define risk, but always remember that they are manifestations of deeper forces in the marketplace.

\textbf{START PAGE 23 AT PRICE ACTION AND MARKET STRUCTURE}
\subsection{Price action and market structure on charts}

\subsection{Charting by hand}

\section{The Market Cycle and the Four Trades}
\subsection{Wyckoff's Market Cycle}
\subsection{The four trades}

\part{Market Structure}
\section{On Trends}
\subsection{The fundamental pattern}
\subsection{Trend structure}
\subsection{A deeper look at pullbacks: the quintessential trend trading pattern}
\subsection{Trend analysis}

\section{On Trading Ranges}
\subsection{Support and resistance}
\subsection{Trading ranges as functional structures}
\section{Interfaces between trends and ranges}
\subsection{Breakout trade: Trading range to trend}
\subsection{Trend to trading range}
\subsection{Trend to opposite trend(Trend reversal)}
\subsection{Trend to same trend (Failure of trend reversal)}

\part{Trading Strategies}
\section{Practical Trading Templates}
\subsection{Failure test}
\subsection{Pullback, buying support or shortening resistance}
\subsection{Pullback, entering lower time frame breakout}
\subsection{Trading complex pullbacks}
\subsection{The anti}
\subsection{Breakouts, entering in the preceding base}
\subsection{Breakouts, entering on first pullback following}
\subsection{Failed breakouts}

\section{Tools for Confirmation}
\subsection{The moving average - the still center}
\subsection{Channels: Emotional extremes}
\subsection{Indicators: MACD}
\subsection{Multiple time frame analysis}

\section{Trade Management}
\subsection{Placing the initial stop}
\subsection{Setting price targets}
\subsection{Active management}
\subsection{Portfolio considerations}
\subsection{Practical issues}

\section{Risk management}
\subsection{Risk and portfolio sizing}
\subsection{Theoretical perspectives on risk}
\subsection{Misunderstood risk}
\subsection{Practical risks in trading}

\part{The Individual, Self-Directed Trader}
\section{The Trader's Mind}
\subsection{Psychological Challenges of the Marketplace}
\subsection{Evolutionary adaptations}
\subsection{Congitive biases}
\subsection{The random reinforcement problem}
\subsection{Emotions: the enemy within}
\subsection{Intuition}
\subsection{Flow}
\subsection{Practical psychology}

\section{Becoming a Trader}
\subsection{The process}
\subsection{Record keeping}
\subsection{Statistical analysis of trading results}
\end{document}
