\documentclass[10pt,twocolumn]{article}
\title{The Art and Science of Technical Analysis}
\begin{document}
\maketitle
\tableofcontents

\part{The Foundation of Technical Analysis}
\section{The Trader's Edge}
Use both \textit{technical} and \textit{fundamental} factors when trading.

Price represents the end product of the analysis and decision making of all market participants, and believe that a careful analysis of price movements can sometimes reveal areas of market imbalance that can offer opportunities for superior risk- adjusted profits.
\subsection{Defining a trading edge}
\begin{itemize}
  \item Excellent execution, risk management, discipline, and proper psychology are all important elements of a good trading plan, but it is all futile if the trading system does not have a positive expectancy.
  \item In all cases, the trading problem reduces to a matter of identifying when a statistical edge is present in the market, acting accordingly, and avoiding market environments that are more random. To do this well, it is essential to have a good understanding of how markets move and also some of the math behind expectancy and probability theory.

\end{itemize}

\subsubsection{Expected Value}
\begin{itemize}
  \item A positive expectancy results when the trader successfully identifies those moments where markets are slightly less random than usual, and places trades that are aligned with the slight statistical edges present in those areas.
  \item If you are not trading with a statistical advantage over the market, everything else is futile.
  \item From a statistical standpoint, the definition of an edge is simple: can you properly identify entry and exit points in the market so that, over a large sample size, the sum of the profit and loss (P\&L) from your winning trades is greater than the sum of your losing trades? The question then becomes: how do you find, develop, refine, and maintain an edge? There are many answers to that question; this book shows one possible path.
\end{itemize}

\subsubsection{Where Does the Edge Come From?}
\begin{itemize}
  \item In an idealized, mathematical random walk world, price would have no memory of where it has been in the past; but in the real world, prices are determined by traders making buy and sell decisions at specific times and prices. When markets revisit these specific prices, the market does have a memory, and we frequently see nonrandom action on these retests of important price levels. People remember the hopes, fears, and pain associated with price extremes. In addition, most large-scale buying follows a more or less predictable pattern: traders and execution algorithms alike will execute part of large orders aggressively, and then will wait to allow the market to absorb the action before resuming their executions. The more aggressive the buyers, the further they will lift offers and the less they will wait between spurts of buying. This type of action, and the memory of other traders around previous inflections, creates slight but predictable tendencies in prices.
  \item The conclusion is logical and unavoidable: buying and selling pressure must, by necessity, leave patterns in the market. Our challenge is to understand how psychology can shape market structure and price action, and to find places where this buying and selling pressure creates opportunities in the form of nonrandom price action.
\end{itemize}

\subsubsection{The Holy Grail}
\textbf{\textit{Every edge we have, as technical traders, comes from an imbalance of buying and selling pressure.}}

If we realize this and if we limit our involvement in the market to those points where there is an actual imbalance, then there is the possibility of making profits. We can sometimes identify these imbalances through the patterns they create in prices, and these patterns can provide actual points around which to structure and execute trades.


\subsection{Finding and developing your edge}


\subsection{General principles of chart reading}
\subsection{Indicators}
\subsection{The two forces: toward a new understanding of market action}
\subsection{Price action and market structure on charts}
\subsection{Charting by hand}

\section{The Market Cycle and the Four Trades}
\subsection{Wyckoff's Market Cycle}
\subsection{The four trades}

\part{Market Structure}
\section{On Trends}
\subsection{The fundamental pattern}
\subsection{Trend structure}
\subsection{A deeper look at pullbacks: the quintessential trend trading pattern}
\subsection{Trend analysis}

\section{On Trading Ranges}
\subsection{Support and resistance}
\subsection{Trading ranges as functional structures}
\section{Interfaces between trends and ranges}
\subsection{Breakout trade: Trading range to trend}
\subsection{Trend to trading range}
\subsection{Trend to opposite trend(Trend reversal)}
\subsection{Trend to same trend (Failure of trend reversal)}

\part{Trading Strategies}
\section{Practical Trading Templates}
\subsection{Failure test}
\subsection{Pullback, buying support or shortening resistance}
\subsection{Pullback, entering lower time frame breakout}
\subsection{Trading complex pullbacks}
\subsection{The anti}
\subsection{Breakouts, entering in the preceding base}
\subsection{Breakouts, entering on first pullback following}
\subsection{Failed breakouts}

\section{Tools for Confirmation}
\subsection{The moving average - the still center}
\subsection{Channels: Emotional extremes}
\subsection{Indicators: MACD}
\subsection{Multiple time frame analysis}

\section{Trade Management}
\subsection{Placing the initial stop}
\subsection{Setting price targets}
\subsection{Active management}
\subsection{Portfolio considerations}
\subsection{Practical issues}

\section{Risk management}
\subsection{Risk and portfolio sizing}
\subsection{Theoretical perspectives on risk}
\subsection{Misunderstood risk}
\subsection{Practical risks in trading}

\part{The Individual, Self-Directed Trader}
\section{The Trader's Mind}
\subsection{Psychological Challenges of the Marketplace}
\subsection{Evolutionary adaptations}
\subsection{Congitive biases}
\subsection{The random reinforcement problem}
\subsection{Emotions: the enemy within}
\subsection{Intuition}
\subsection{Flow}
\subsection{Practical psychology}

\section{Becoming a Trader}
\subsection{The process}
\subsection{Record keeping}
\subsection{Statistical analysis of trading results}
\end{document}
