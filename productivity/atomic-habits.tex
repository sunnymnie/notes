\documentclass[10pt,twocolumn]{../notes}
\title{Atomic Habits}
\begin{document}
\maketitle
\tableofcontents
\section{The Fundamentals}
\Advice To make earth-shattering improvements, do not seek for massive changes, but change by just making small improvements on a daily basis
\Quote If you want to predict where you’ll end up in life, all you have to do is follow the curve of tiny gains or tiny losses, and see how your daily choices will compound ten or twenty years down the line

\Advice Be patient with your habits
\Fact The effect of small habits are delayed. You don’t immediately see good results from good small habits, and neither bad results from small bad habits. But, overtime, they compound to something much larger
\Quote Your outcomes are a lagging measure of your habits
\Fact Why people stop building positive habits: They see that they’re failing to make a tangible result and decide to quit
\Quote It is not until months or years later that we realize the true value of the previous work we have done
\Quote Small changes often appear to make no difference until you cross a critical threshold. The most powerful outcomes of any compounding process are delayed. You need to be patient

\Image{the-plateau}{the plateau of latent potential}

\Advice Focus not only on making good habits, but also removing bad ones
\Quote Habits are a double-edged sword. They can work for you or against you
\begin{List}{Positive Compounding}
\item Productivity compounds: Accomplishing one extra task every day or automating an old task every day can lead to a lot over an entire career.
\item Knowledge compounds: Learning one new idea every day not only makes you know more, but allows you to see old ideas through new perspectives.
\item Relationships compound: Helping others more and being nicer in each interaction can result in a network of strong connections over time.
\end{List}

\begin{List}{Negative Compounding}
\item Stress compounds: Stressing every day can lead to serious health issues
\item Negative thoughts compound: The more you think of yourself or others negatively, you will spiral uncontrollably towards that direction.
\end{List}

\Advice Focus on systems rather than setting goals.
\Fact Society worries about setting specific, actionable goals, but the real results lie when you focus on the system.
\Quote You do not rise to the level of your goals. You fall to the level of your systems

\section{How Your Habits Shape Your Identity (and Vice Versa)}
\Image{opi-chart}{Three layers of behaviour change}




\end{document}
