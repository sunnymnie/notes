\documentclass[10pt,twocolumn]{../notes}
\title{Atomic Habits}
\begin{document}
\maketitle
\tableofcontents
\section{The Fundamentals}
\Advice To make earth-shattering improvements, do not seek for massive changes, but change by just making small improvements on a daily basis
\Quote If you want to predict where you’ll end up in life, all you have to do is follow the curve of tiny gains or tiny losses, and see how your daily choices will compound ten or twenty years down the line

\Advice Be patient with your habits
\Fact The effect of small habits are delayed. You don’t immediately see good results from good small habits, and neither bad results from small bad habits. But, overtime, they compound to something much larger
\Quote Your outcomes are a lagging measure of your habits
\Fact Why people stop building positive habits: They see that they’re failing to make a tangible result and decide to quit
\Quote It is not until months or years later that we realize the true value of the previous work we have done
\Quote Small changes often appear to make no difference until you cross a critical threshold. The most powerful outcomes of any compounding process are delayed. You need to be patient

\Image{the-plateau}{the plateau of latent potential}

\Advice Focus not only on making good habits, but also removing bad ones
\Quote Habits are a double-edged sword. They can work for you or against you
\begin{List}{Positive Compounding}
\item Productivity compounds: Accomplishing one extra task every day or automating an old task every day can lead to a lot over an entire career.
\item Knowledge compounds: Learning one new idea every day not only makes you know more, but allows you to see old ideas through new perspectives.
\item Relationships compound: Helping others more and being nicer in each interaction can result in a network of strong connections over time.
\end{List}

\begin{List}{Negative Compounding}
\item Stress compounds: Stressing every day can lead to serious health issues
\item Negative thoughts compound: The more you think of yourself or others negatively, you will spiral uncontrollably towards that direction.
\end{List}

\Advice Focus on systems rather than setting goals.
\Fact Society worries about setting specific, actionable goals, but the real results lie when you focus on the system.
\Quote You do not rise to the level of your goals. You fall to the level of your systems

\section{How Your Habits Shape Your Identity (and Vice Versa)}
\Image{opi-chart}{Three layers of behaviour change}
\Bold{Outcomes:} Concerned with results (ex: publishing book, general goals)
\Bold{Process:} Concerned with changing your habits and systems (ex: new gym routine)
\Bold{Identity:} Changing your beliefs, worldview, self-image, and judgements

\Advice Focus on who you wish to become rather than what you want to achieve
\Example Being offered a cigarette and saying “No thanks. I’m trying to quit” vs. “No thanks. I’m not a smoker.” Small difference but one is a goal and the other is a shift in identity
\Quote The ultimate form of intrinsic motivation is when a habit becomes part of your identity
\Fact Once your pride gets involved in your habit, you’ll fight hard to maintain them
\Quote True behaviour change is identity change. You might start a habit because of motivation, but the only reason you’ll stick with one is that it becomes part of your identity
\Example The goal is not to read a book, the goal is to become a reader.
\Example The goal is not to run a marathon, the goal is to become a runner.
\Example The goal is not to learn an instrument, the goal is to become a musician.
\Fact The reason why your positive habits fail to stick is because they conflict with your current identity
\Fact Feedback loops: Your habits shape your identity, and your identity shapes your habits

\Advice Know what result you want, work backward to find what type of person could get those results. Work to become that person by continuously accumulating small habits
\begin{Olist}{Two step process to changing your identity}
  \item Decide the type of person you want to be
  \item Prove it to yourself with small wins
\end{Olist}
\Quote The most effective way to change your habits is to focus not on what you want to achieve, but on who you wish to become
\Quote Every action you take is a vote for the type of person you wish to become. No single instance will transform your beliefs, but as the votes build up, so does the evidence of your new identity
\Quote The most practical way to change who you are is to change what you do

\Image{habit-stages-chart}{the science of how habits work}
\begin{Olist}{}
  \item Cue: Triggers brain to initiate a behaviour
  \item Craving: Motivational force behind every habit. Every craving is linked to a desire to change your internal state
  \item Response: The actual habit performed. If friction is low and the reward is great, the response will be carried out
  \item Reward: End goal of every habit. “The cue is about noticing the reward. The craving is about wanting the reward. The response is about obtaining the reward
\end{Olist}
\Fact We chase rewards because they satisfy us and because they teach us that these actions are worth remembering in the future.
\Fact If a behaviour is insufficient in any of the four stages, it will not become a habit

\section{The 1st Law: Make it Obvious}
\Advice Create a habits scorecard to observe what is going on
\Quote This is one of the most surprising insights about our habits: you don’t need to be aware of the cue for a habit to begin. You can notice an opportunity and take action without dedicating conscious attention to it. This is what makes habits useful.
\Fact Many of our habits are performed subconsciously
\Quote Once our habits become automatic, we stop paying attention to what we are doing.
\Quote The process of behaviour change always starts with awareness. You need to be aware of your habits before you can change them.
\begin{Olist}{How to create a habits scorecard}
  \item List out all of your daily habits
  \item Looking at each habit, ask yourself if that habit is a good, bad, or neutral habit. Mark each habit on the list as such.
  \item If a habit is hard to determine with step 2, ask yourself: “Does this behaviour help me become the type of person I wish to be? Does this habit cast a vote for or against my desired identity?”
\end{Olist}
\Advice Create an implementation intention: “I will BEHAVIOUR at TIME in LOCATION”
\Example Studying. I will study Spanish for twenty minutes at 6 p.m. in my bedroom
\Example Exercise. I will exercise for one hour at 5 p.m. in my local gym
\Example Marriage. I will make my partner a cup of tea at 8 a.m. in the kitchen
\Fact The two most important cues that can trigger a habit are time and place
\Quote People who make a specific plan for when and where they will perform a new habit are more likely to follow through






\end{document}
