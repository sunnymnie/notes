\documentclass[10pt,twocolumn]{../notes}
\title{The Art and Science of Technical Analysis}
\begin{document}
\maketitle
\tableofcontents
\section{The Foundation of Technical Analysis}
\subsection{The Trader’s Edge}

\Advice A positive expectancy results when the trader successfully identifies moments where markets are slightly less random than usual, and places trades that are aligned with the slight statistical edges present in those areas.
\Fact Most of the time, prices fluctuate in a more or less random fashion.
\Quote Though a trader may make some profitable trades in this type of environment purely due to random chance, it is simply not possible to profit in the long run; nothing the trader can do will have a positive effect on the bottom line as long as randomness dominates price changes.

\Quote Some traders are drawn to focus on high-probability (high win rate) trading, while others focus on finding trades that have excellent reward/risk profiles. Neither of these approaches is better than the other; what matters is how these two factors of probability and reward/risk ratio interact.

\Advice If you are not trading with a statistical advantage over the market, everything else is futile. Nothing will help. Discipline, money management, execution skills, and positive thinking add great value in support of an actual edge, but they are not edges in themselves.
\Fact Expected value: Formally, for $k$ possible scenarios, each with a payoff of $x$ and associated probability $p$, the expected value $E()$ is defined as: $E(x) = x_1p_1+x_2p_2+...+x_kp_k$ or $E(x) = \sum_{i=1}^{k} x_ip_i$
\Quote From a statistical standpoint, the definition of an edge is simple: can you properly identify entry and exit points in the market so that, over a large sample size, the sum of the profit and loss (P\&L) from your winning trades is greater than the sum of your losing trades?

\subsubsection{Where Does the Edge Come From?}
\Quote One of the assumptions of academic finance is that people make rational decisions in their own best interests, after carefully calculating the potential gains and losses associated with all possible scenarios. This may be true at times, but not always.
\Quote The market does not simply react to new information flow; it reacts to that information as it is processed through the lens of human emotion.

\Quote In an idealized, mathematical random walk world, price would have no memory of where it has been in the past; but in the real world, prices are determined by traders making buy and sell decisions at specific times and prices. When markets revisit these specific prices, the market does have a memory, and we frequently see nonrandom action on these retests of important price levels. People remember the hopes, fears, and pain associated with price extremes.
\Quote In addition, most large-scale buying follows a more or less predictable pattern: traders and execution algorithms alike will execute part of large orders aggressively, and then will wait to allow the market to absorb the action before resuming their executions. The more aggressive the buyers, the further they will lift offers and the less they will wait between spurts of buying. This type of action, and the memory of other traders around previous inflections, creates slight but predictable tendencies in prices.
\Quote The conclusion is logical and unavoidable: buying and selling pressure must, by necessity, leave patterns in the market. Our challenge is to understand how psychology can shape market structure and price action, and to find places where this buying and selling pressure creates opportunities in the form of nonrandom price action.

\end{document}
